%%%%%%%%%%%%%%%%%%%%%%%%%%%%%%%%%%%%%%%%%
% Beamer Presentation
% LaTeX Template
% Version 1.0 (10/11/12)
%
% This template has been downloaded from:
% http://www.LaTeXTemplates.com
%
% License:
% CC BY-NC-SA 3.0 (http://creativecommons.org/licenses/by-nc-sa/3.0/)
%
%%%%%%%%%%%%%%%%%%%%%%%%%%%%%%%%%%%%%%%%%

%----------------------------------------------------------------------------------------
%	PACKAGES AND THEMES
%----------------------------------------------------------------------------------------

\documentclass{beamer}
%\newtheorem{problem}{Problem}
%\setbeamertemplate{footline}[frame number]{}
\setbeamertemplate{navigation symbols}{}

\mode<presentation> {

% The Beamer class comes with a number of default slide themes
% which change the colors and layouts of slides. Below this is a list
% of all the themes, uncomment each in turn to see what they look like.

%\usetheme{default}
%\usetheme{AnnArbor}
%\usetheme{Antibes}
%\usetheme{Bergen}
%\usetheme{Berkeley}
%\usetheme{Berlin}
%\usetheme{Boadilla}
%\usetheme{CambridgeUS}
%\usetheme{Copenhagen}
%\usetheme{Darmstadt}
%\usetheme{Dresden}
%\usetheme{Frankfurt}
%\usetheme{Goettingen}
%\usetheme{Hannover}
%\usetheme{Ilmenau}
%\usetheme{JuanLesPins}
%\usetheme{Luebeck}
\usetheme{Madrid}
%\usetheme{Malmoe}
%\usetheme{Marburg}
%\usetheme{Montpellier}
%\usetheme{PaloAlto}
%\usetheme{Pittsburgh}
%\usetheme{Rochester}
%\usetheme{Singapore}
%\usetheme{Szeged}
%\usetheme{Warsaw}

%\DeclarePairedDelimiter\ceil{\lceil}{\rceil}
%\DeclarePairedDelimiter\floor{\lfloor}{\rfloor}

\newcommand{\minpol}{\textnormal{MinPoly}_{\mathbb{F}}}

% As well as themes, the Beamer class has a number of color themes
% for any slide theme. Uncomment each of these in turn to see how it
% changes the colors of your current slide theme.

%\usecolortheme{albatross}
%\usecolortheme{beaver}
%\usecolortheme{beetle}
%\usecolortheme{crane}
%\usecolortheme{dolphin}
%\usecolortheme{dove}
%\usecolortheme{fly}
%\usecolortheme{lily}
%\usecolortheme{orchid}
%\usecolortheme{rose}
%\usecolortheme{seagull}
%\usecolortheme{seahorse}
%\usecolortheme{whale}
%\usecolortheme{wolverine}
\newcommand{\N}{\mathbb{N}}
\newcommand{\K}{\mathbb{K}}
%\newcommand{\L}{\mathbb{L}}
\newcommand{\F}{\mathbb{F}}

%\setbeamertemplate{footline} % To remove the footer line in all slides uncomment this line
%\setbeamertemplate{footline}[page number] % To replace the footer line in all slides with a simple slide count uncomment this line

%\setbeamertemplate{navigation symbols}{} % To remove the navigation symbols from the bottom of all slides uncomment this line
}

\usepackage{graphicx} % Allows including images
\usepackage{booktabs} % Allows the use of \toprule, \midrule and \bottomrule in tables
\usepackage{mathtools}
\usepackage{amsmath}
%\newcommand\keq{\stackrel{\mathclap{\mbox{\tiny k-uni}}}{=}}
\newcommand{\f}{\mathbb{F}}
\newcommand{\ot}{\widetilde{O}}

%----------------------------------------------------------------------------------------
%	TITLE PAGE
%----------------------------------------------------------------------------------------

\title[]{Computing the Characteristic Polynomial of a Finite Rank Two Drinfeld Module} % The short title appears at the bottom of every slide, the full title is only on the title page

\author{Yossef Musleh} % Your name
\institute[UW] % Your institution as it will appear on the bottom of every slide, may be shorthand to save space
{
University of Waterloo \\ % Your institution for the title page
\medskip
\textit{ymusleh@uwaterloo.ca} % Your email address
}
\date{\today} % Date, can be changed to a custom date

\begin{document}

\begin{frame}
\titlepage % Print the title page as the first slide
\end{frame}

%\begin{frame}
%\frametitle{Overview} % Table of contents slide, comment this block out to remove it
%\tableofcontents % Throughout your presentation, if you choose to use \section{} and \subsection{} commands, these will automatically be printed on this slide as an overview of your presentation
%\end{frame}

%----------------------------------------------------------------------------------------
%	PRESENTATION SLIDES
%----------------------------------------------------------------------------------------

\begin{frame}
\frametitle{Motivation}

\begin{itemize}

\item  Elliptic Curves: Important to Classical Algebraic Geometry and Number Theory

\begin{itemize}
    \item Fermat's Last Theorem
    \item Birch and Swinnerton-Dyer Conjecture
    \item Elliptic Curve Cryptography
    
\end{itemize}

\item Drinfeld Modules
\begin{itemize}
\item Rank 2 case a "function field analogue of elliptic curves"
    \item Used to prove special cases of Langlands Conjectures [Drinfeld, 1974]
    \item Used in polynomial factorization algorithms over finite fields
    
 
    \item Cryptography over Drinfeld modules - insecure [Scanlon, 2001]
\end{itemize}
  
  
  
  
\end{itemize}

\end{frame}



\begin{frame}
\frametitle{Background and Notation}

\begin{itemize}

\item $\mathbb{L} = \mathbb{F}_q[T]/f(T)$
\item deg$(f) = n$
\item $\gamma: \mathbb{F}_q \to \mathbb{L}$ finite fields
\item $\sigma(x) := x^q$
\item $\mathbb{L}[X,\sigma] := $ skew polynomials over $\mathbb{L}$ subject to $Xa = \sigma(a)X$ for $a \in \mathbb{L}$ 

\end{itemize}
\end{frame}
\begin{frame}{Background and Notation}

%\begin{example}
%Let $\mathbb{F} = \f_2$, \mathbb{L} = \f_2[x]/(x^2 + x + 1)$, %$\sigma(x) = x^2$
%\[ a:= (x + 1)X^2 + xX + x + 1 \]
%\[b := X\]
%\[ab = (x + 1)X^3 + xX^2 + (x + 1)X\]
%\[ba = xX^3 + (x+1)X^2 + xX \]
%\end{example}

\begin{example}
Let $q = 2$, $\mathbb{L} = \mathbb{F}_2[T]/(T^2 + T + 1) = \mathbb{F}_4$, $\sigma(T) = T^2$
\[ a:= (T + 1)X^2 + TX + T + 1 \]
\[b := X\]
\[ab = (T + 1)X^3 + TX^2 + (T + 1)X\]
\[ba = TX^3 + (T+1)X^2 + TX \]
\end{example}



\end{frame}

%------------------------------------------------


%-------------------------------------


%\begin{frame}\frametitle{Drinfeld Modules: Preliminaries}


%\end{frame}




%-------------------------------------

\begin{frame}
\frametitle{Drinfeld Modules}

\begin{definition}
A \textbf{Drinfeld Module} is a ring homomorphism $\varphi: \mathbb{F}_q[x] \to \mathbb{L}[X,\sigma]$ such that 

\begin{itemize}
    \item $\varphi(x) = \gamma(x) + a_1X + \ldots + a_rX^r$ with $a_r \neq 0$ for some $r \geq 1$
\end{itemize}
\end{definition}

\begin{itemize}
    \item The value $r$ is referred to as the \textit{rank} of the Drinfeld Module
        \item In the rank-2 case we say $\varphi = (g, \Delta)$ with $\varphi(x) = \gamma(x) + gX + \Delta X^2$
        \item Identify elements of $\mathbb{L}[X,\sigma]$ with endomorphisms of $\mathbb{L}$
        \item $\varphi_c := \varphi(c)$
 
 \end{itemize}
 \end{frame}
 
 \begin{frame}{Drinfeld Modules}
   \begin{example}
   Let $q =2$, $\mathbb{F}_2[T]/(T^2 + T + 1)$.
   \[ \f_2[T]/(T^2 + T + 1) \cong \f_4 \subset \f_{16}\]
   \[ \gamma : f(T) \mapsto f(T) \mod T^2 + T + 1  \]
   \[ \varphi_T := T + X + X^2 \]
   \[ \varphi = (1,1)\]
   \end{example}

\end{frame}





%-----------------------------------

\begin{frame}
\frametitle{Point Counting}

\begin{itemize}
\item Classical problem: given an elliptic curve $E$, find the number of points over some finite field $\mathbb{F}_q$ 
\item Schoof gave the first polynomial time algorithm
\item Hasse's theorem provides a bound $ | |E(\mathbb{F}_q)| - q - 1  | \leq 2 \sqrt{q} $

\item Computing the LHS above reduces to computing the characteristic polynomial of the Frobenius endomorphism

\item Goal: Translate this to the rank-2 Drinfeld Module setting
\end{itemize}

\end{frame}


%-------------------------------------


%-------------------------------------




%--------------------------------------------------

\begin{frame}
\frametitle{Point Counting}

\begin{theorem}[Gekeler, 1991]
Let $\varphi$ be a rank-2 Drinfeld Module, and let $\tau = \sigma^n$. Then there is a polynomial, the \textbf{characteristic polynomial} of $\varphi$,  $Y^2 - aY +b \in \mathbb{F}_q[Y]$ such that

\[\tau^2 -\phi_a\tau + \phi_b = 0\]
in $\mathbb{L}[X,\sigma]$
\end{theorem}

\begin{itemize}
    %\item The \textit{Characteristic Polynomial} of $\phi$
    \item $a$ is the \textit{Frobenius Trace}, $b$ the \textit{Frobenius Norm}
    \item $\deg(a) \leq \frac{n}{2}$, $\deg(b) \leq n$
\end{itemize}

\end{frame}


\begin{frame}
\frametitle{Point Counting}

\begin{problem}
Given a rank-2 Drinfeld module $\varphi = (g,\Delta)$, compute its Frobenius Trace and Norm
\end{problem}

\end{frame}

%-------------------------------------------------

%\begin{frame}{}
%    \begin{example}
%$\f = \f_2$, $L = \f_{16}$, $\gamma = \textnormal{quotient by } T^2 + T + 1$, $\phi = (1,1)$

%\[b = T^4 + T^2 + 1\]
%\[a = T^2 + T\]
%\[\phi_{T}^2 = X^4 + (T^2 + T)X + T^2\]
%\[\phi_T^4 = X^8 + X^2 + T + 1\]
%\[X^8 + \phi_aX^4 +  \phi_b\]
%\[X^8 - (X^4 + X^2 + (T^2 + T + 1)X + T^2 + T)X^4 + X^8 + X^4 + X^2 (T^2 + T)X + T^2 + T + 1

%\]
%\end{example}
%\end{frame}




%-----------------------------------------------


\begin{frame}
\frametitle{Main Result}

\begin{itemize}
\item Computing the Frobenius norm is relatively straightforward [Gekeler, 1991]
\item Computing the Frobenius trace turns out to be much harder

\begin{theorem}
In a RAM model, one can compute the Frobenius trace of a rank 2 Drinfeld module
\begin{enumerate}
\item in Monte Carlo time $O\tilde{~}(n^2 \log^2 q)$
\item in deterministic time $(n^2 \log q + n \log^2 q)^{1+o(1)}$

\end{enumerate}
\end{theorem}

\end{itemize}


\end{frame}

%------------------

\begin{frame}
\frametitle{Known Results}

\begin{itemize}
\item Modular composition: $F(G) \mod H$ , $\deg F, G,H \leq d$
\begin{itemize}
    \item Kedlaya-Umans: $(n^{1 + \varepsilon}
\log q)^{1+o(1)}$
    \item Lecerf-van der Hoeven: $(n
\log q)^{1+o(1)}$
\end{itemize}
\item Degree at most $d$ skew polynomial multiplication: $(d^{1.67} n\log q)^{1+o(1)}$
\item Naive Frobenius
    
\end{itemize}


\end{frame}


%-----------------



\begin{frame}
\frametitle{Previous Techniques}

\begin{itemize}
    \item $[$Gekeler, 2008] gives a straightforward approach
    \item Set $\phi_{T^i} = \sum_{j=0}^{2i}f_{i,j} X^j$, $f_{i,j} \in L$
    \item Recall: $\tau^2 - \phi_a\tau + \phi_b = 0$
    \item Construct a triangular system for the coefficients of $a$ in terms of $f_{i,j}$
    \item Compute $f_{i,j}$ using a recurrence
    \item Overall runtime $(n^3 \log q + n\log^2 q)^{1 + o(1)}$
\end{itemize}


\end{frame}


%-----------------------------------------------



%-----------------------------------------------



%-------------------------------------------------

\begin{frame}{Previous Techniques}
\begin{itemize}

\item A faster algorithm based on the Hasse invariant exists when $\gamma$ is an automorphism
\item Narayanan gave a randomized algorithm when $q$ is odd and  CharPoly$(\varphi_x) = $ MinPoly$(\varphi_x)$
\item Based on the automorphism projection algorithm of [Kaltofen and Shoup, 1998]
\item Runs in Monte Carlo time $(n^{1.885} \log q + n \log^2 q)^{1+o(1)}$,
  if $ \deg \textnormal{MinPoly}(\varphi_T) = n$
\item Includes an unjustified claim which we repair using a block Wiedemann approach
\end{itemize}
\end{frame}

%------------------------------------------------



%------------------------------------------------

\begin{frame}{A Randomized Algorithm}

\begin{itemize}
    \item Inspired by Shoup  
    \item Recall: $\tau^2 + \varphi_b = \varphi_a \tau$
    \item Choose random $\alpha \in \mathbb{L}$, $\ell : \mathbb{L} \to \f_q$
    \item Identify $\tau \mapsto \sigma^n$, $r := \alpha + \varphi_b(\alpha) = \varphi_a(\alpha)$
    \item $a := \sum_{i=0}^{\left\lfloor \frac{n}{2} \right\rfloor}a_iT^i$
    \item For $j \geq 0$: $\ell(\varphi_T^j(r)) = \sum_{i = 0}^{\left\lfloor{\frac{n}{2}} \right\rfloor}a_i\ell(\varphi_T^{i+j}(\alpha))$
    \end{itemize}
    
    \end{frame}
    
    \begin{frame}{A Randomized Algorithm}
    \begin{itemize}
    \item For a choice of $\kappa$, we can construct a Hankel system
\end{itemize}
\[ \begin{bmatrix}\ell(\alpha) & \ell(\varphi_T(\alpha)) & \ldots & \ell(\varphi_T^{\left\lfloor n/2 \right\rfloor}(\alpha)) \\ \vdots & \vdots & & \vdots \\ 

\ell(\varphi_T^{j}(\alpha)) & \ell(\varphi_T^{1+j}(\alpha)) & \ldots & \ell(\varphi_T^{\left\lfloor n/2 \right\rfloor+j}(\alpha)) \\ \vdots & \vdots & & \vdots \\

\ell(\varphi_T^{\kappa}(\alpha)) & \ell(\varphi_T^{1 + \kappa }(\alpha)) & \ldots & \ell(\varphi_T^{\left\lfloor n/2 \right\rfloor + \kappa}(\alpha))

\end{bmatrix} \begin{bmatrix} a_0 \\ a_1 \\ \vdots \\ a_i \\ \vdots \\ a_{\left\lfloor n/2 \right\rfloor} \end{bmatrix} = \begin{bmatrix} \ell(r) \\ \ell(\varphi_T(r)) \\ \vdots \\ \ell(\varphi_T^j(r)) \\ \vdots  \\   \ell(\varphi_T^{\kappa}(r)) \end{bmatrix} \]
    
\end{frame}

%------------------------------------------------


%-------------------------------------------------


\begin{frame}{A Randomized Algorithm}

\begin{itemize}
    \item  With probability at least $(1 - \frac{n}{2q})^2$ we have that $\minpol(\{\ell(\varphi_T^i(\alpha)\}_i) = \minpol(\varphi_T)$
    \item $\frac{n-1}{2} \leq \deg \minpol(\varphi_T) \leq n$
    \item Take $\kappa = \deg \minpol(\varphi_T)$
    \item After possibly augmenting the Hankel matrix an upper left submatrix of size at least $\left\lfloor \frac{n}{2} \right\rfloor + 1$ is invertible in almost all cases
\end{itemize}
    
\end{frame}


%-------------------------------------------------

\begin{frame}{A Randomized Algorithm}
    \begin{itemize}
        \item Takes $n^2 \log q$ $\f_q$ operations to compute all entries of the Hankel matrix
    \item $n$ to solve the Hankel System
    \item Overall bit complexity of $O\tilde{~}(n^2 \log^2 q)$
   % \item Overall bit complexity 
    \end{itemize}
\end{frame}

%-------------------------------------------------

\begin{frame}{A Deterministic Algorithm}

\begin{itemize}
    \item Inspired by Schoof's algorithm for elliptic curves
    \item Exploit the ring-homomorphic properties of $\varphi$
    \item Suppose $ \frac{n}{2} + 1 < q$
    \item Pick a set $\{e_0, \ldots e_{\frac{n}{2}}\} \subset \mathbb{F}$
    \item Characteristic equation: $a(e_i) X^n  = X^{2n} + b(e_i) \mod \varphi_{x} - e_i $
    \item Find $a(e_i)$ and interpolate
\end{itemize}
    
\end{frame}


%-------------------------------------------------

\begin{frame}{A Deterministic Algorithm}
\begin{itemize}
    \item Want to compute $X^n \mod \varphi_{T} - e_i$
    \item Define $X^j \mod \varphi_{T} - e_i := \nu_j + \mu_j X $ with $\nu_j, \mu_j \in \mathbb{L}$
    %\item $\phi_T - e_i = \gamma_T - e_i + gX + \Delta X^2$
    %\item 
    %\item We obtain the recurrences $\nu_{j+1} = -\frac{\gamma_x - e_i}{\Delta}\mu_{j}^q$ and $\mu_{j + 1} = \nu_j^q - \frac{g}{\Delta} \mu_j^q$
    
    \item Set $\alpha := -\frac{\gamma_x - e_i}{\Delta}$, $\beta := - \frac{g}{\Delta}$, $M^{(q^j)} := \begin{bmatrix} 0 & \alpha^{q^j} \\ 1 & \beta^{q^j} \end{bmatrix}$
    
    \item Simple recurrence gives $\begin{bmatrix} \nu_{n} \\ \mu_n  \end{bmatrix} = M M^{(q)} \ldots M^{(q^{n-1})}  \begin{bmatrix} 1 \\ 0  \end{bmatrix}$
    
    \item Modular composition and a recursive procedure computes the expression 
    
    
    
\end{itemize}
    
\end{frame}

%------------------------------------------------

\begin{frame}{A Deterministic Algorithm}
\begin{itemize}
    \item Given $\nu_n, \mu_n$ we can use the characteristic equation to compute $a(e_i)$
        \item If $\mu_n \neq 0$, $a(e_i) = \nu_n + \nu_n^q + \mu_n^q \beta$, otherwise $a(e_i) = \nu_n + b(e_i)$
    \item Overall runtime is $(n^2 \log q + n \log^2 q)^{1+o(1)}$ bit operations

    \item This approach can be extended to the cases where $\frac{n}{2} + 1 \geq q$ by using $\varphi_{g}$ for irreducible $g$
\end{itemize}
    
\end{frame}

%-------------------------------------------------


%-------------------------------------------------

%\begin{frame}
%\begin{example}
%Let $q = 5$, $n = 4$, $L = \mathbb{F}_5[T]/(T^4 + 4T^2 + 4T + 2)$, $\phi = (1,1)$. Then for $e_0 = 0$ we have $\alpha = 4t$, $\beta = 4$. Letting

%\[M = \begin{bmatrix}0 & 4T \\ 1 & 4 \end{bmatrix}\]

%We get:

%\[M M^{(5)} M^{(25)} M^{(125)} = \begin{bmatrix} T^3 + T^2 + 3 & T^3 + 2T + 3 \\ 2T^3 + 2T^2 + 3T + 4 & 4T^3 + 4T^2 + 4 \end{bmatrix}\]

%So $\nu_4 = T^3 + T^2 + 3$ and $\mu_4 = 2T^3 + 2T^2 + 3T + 4$ and

%\[a(0) = \nu_4 + \nu_4^5 + 4\mu_4^5 = 2\]


%\end{example}
%\end{frame}

%\begin{frame}

%\begin{example}
%Repeating for $e_1 = 1$, $e_2 = 2$ we get $a(1) = 3$ and $a(2) = 3$. We interpolate to get

%\[a = (T-1)(T-2) + 2T(T-2) + 4T(T-1)   = 2T^2 + 4T + 2.\]
%\end{example}
    
%\end{frame}

%--------------------------------------------------


\begin{frame}{Conclusion}

\begin{table}[]
\begin{tabular}{|c|c|}
\hline
Algorithm & Runtime  \\ \hline
Gekeler &  $(n^{3}\log
  q + n \log^2 q)^{1+o(1)}$ \\
Narayanan & $(n^{1.885} \log q + n \log^2 q)^{1+o(1)}$  \\
Hasse & $(n^{3/2} \log q + n \log^2 q)^{1+o(1)}$  \\
Randomized & $O\tilde{~}(n^2 \log^2 q)$  \\
Deterministic & $(n^2 \log q + n \log^2 q)^{1+o(1)}$ \\ \hline
\end{tabular}
\end{table}
    
\end{frame}

%-----------------------------------------

%\begin{frame}{Conclusion}

%\begin{table}[]
%\begin{tabular}{|c|c|}
%\hline
%Algorithm & Runtime  \\ \hline
%Gekeler &  $(n^{\theta+2}\log
%  q + n \log^2 q)^{1+o(1)}$ \\
%Narayanan & $(n^{1.885} \log q + n \log^2 q)^{1+o(1)}$  \\
%Hasse & $(n^{\theta+1/2} \log q + n \log^2 q)^{1+o(1)}$  \\
%Randomized & $O\tilde{~}(n^2 \log^2 q)$  \\
%Deterministic & $(n^2 \log q + n \log^2 q)^{1+o(1)} \\ \hline
%\end{tabular}
%\end{table}
    
%\end{frame}

%-------------------------------------------

\begin{frame}{Conclusion}

\begin{itemize}
    \item We provide two new algorithms for computing the characteristic polynomial of a rank-2 Drinfeld module
    \item Sub-cubic runtime in $[L:\f_q]$
    \item Works for all finite Drinfeld modules
\end{itemize}
    
\end{frame}

%--------------------------------------------------

%\begin{frame}
%\frametitle{References}
%\footnotesize{
%\begin{thebibliography}{99} % Beamer does not support BibTeX so references must be inserted manually as below
%\bibitem[Smith, 2012]{p1} M.B. Paterson, D.R. Stinson (2015)
%\newblock Combinatorial Characterizations of algebraic manipulation detection codes involving generalized difference families
%\end{thebibliography}
%}
%\end{frame}

%------------------------------------------------

%\begin{frame}
%\Huge{\centerline{Fin}}
%\end{frame}

%----------------------------------------------------------------------------------------

\end{document} 