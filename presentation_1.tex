%%%%%%%%%%%%%%%%%%%%%%%%%%%%%%%%%%%%%%%%%
% Beamer Presentation
% LaTeX Template
% Version 1.0 (10/11/12)
%
% This template has been downloaded from:
% http://www.LaTeXTemplates.com
%
% License:
% CC BY-NC-SA 3.0 (http://creativecommons.org/licenses/by-nc-sa/3.0/)
%
%%%%%%%%%%%%%%%%%%%%%%%%%%%%%%%%%%%%%%%%%

%----------------------------------------------------------------------------------------
%	PACKAGES AND THEMES
%----------------------------------------------------------------------------------------

\documentclass{beamer}
%\newtheorem{problem}{Problem}

\mode<presentation> {

% The Beamer class comes with a number of default slide themes
% which change the colors and layouts of slides. Below this is a list
% of all the themes, uncomment each in turn to see what they look like.

%\usetheme{default}
%\usetheme{AnnArbor}
%\usetheme{Antibes}
%\usetheme{Bergen}
%\usetheme{Berkeley}
%\usetheme{Berlin}
%\usetheme{Boadilla}
%\usetheme{CambridgeUS}
%\usetheme{Copenhagen}
%\usetheme{Darmstadt}
%\usetheme{Dresden}
%\usetheme{Frankfurt}
%\usetheme{Goettingen}
%\usetheme{Hannover}
%\usetheme{Ilmenau}
%\usetheme{JuanLesPins}
%\usetheme{Luebeck}
\usetheme{Madrid}
%\usetheme{Malmoe}
%\usetheme{Marburg}
%\usetheme{Montpellier}
%\usetheme{PaloAlto}
%\usetheme{Pittsburgh}
%\usetheme{Rochester}
%\usetheme{Singapore}
%\usetheme{Szeged}
%\usetheme{Warsaw}

%\DeclarePairedDelimiter\ceil{\lceil}{\rceil}
%\DeclarePairedDelimiter\floor{\lfloor}{\rfloor}

\newcommand{\minpol}{\textnormal{MinPoly}_{\mathbb{F}}}

% As well as themes, the Beamer class has a number of color themes
% for any slide theme. Uncomment each of these in turn to see how it
% changes the colors of your current slide theme.

%\usecolortheme{albatross}
%\usecolortheme{beaver}
%\usecolortheme{beetle}
%\usecolortheme{crane}
%\usecolortheme{dolphin}
%\usecolortheme{dove}
%\usecolortheme{fly}
%\usecolortheme{lily}
%\usecolortheme{orchid}
%\usecolortheme{rose}
%\usecolortheme{seagull}
%\usecolortheme{seahorse}
%\usecolortheme{whale}
%\usecolortheme{wolverine}
\newcommand{\N}{\mathbb{N}}
\newcommand{\K}{\mathbb{K}}
\newcommand{\L}{\mathbb{L}}
\newcommand{\F}{\mathbb{F}}

%\setbeamertemplate{footline} % To remove the footer line in all slides uncomment this line
%\setbeamertemplate{footline}[page number] % To replace the footer line in all slides with a simple slide count uncomment this line

%\setbeamertemplate{navigation symbols}{} % To remove the navigation symbols from the bottom of all slides uncomment this line
}

\usepackage{graphicx} % Allows including images
\usepackage{booktabs} % Allows the use of \toprule, \midrule and \bottomrule in tables
\usepackage{mathtools}
\usepackage{amsmath}
%\newcommand\keq{\stackrel{\mathclap{\mbox{\tiny k-uni}}}{=}}
\newcommand{\f}{\mathbb{F}}
\newcommand{\ot}{\widetilde{O}}

%----------------------------------------------------------------------------------------
%	TITLE PAGE
%----------------------------------------------------------------------------------------

\title[master talk]{Computing the Characteristic Polynomial of a Finite Rank Two Drinfeld Module} % The short title appears at the bottom of every slide, the full title is only on the title page

\author{Yossef Musleh} % Your name
\institute[UW] % Your institution as it will appear on the bottom of every slide, may be shorthand to save space
{
University of Waterloo \\ % Your institution for the title page
\medskip
\textit{ymusleh@uwaterloo.ca} % Your email address
}
\date{\today} % Date, can be changed to a custom date

\begin{document}

\begin{frame}
\titlepage % Print the title page as the first slide
\end{frame}

%\begin{frame}
%\frametitle{Overview} % Table of contents slide, comment this block out to remove it
%\tableofcontents % Throughout your presentation, if you choose to use \section{} and \subsection{} commands, these will automatically be printed on this slide as an overview of your presentation
%\end{frame}

%----------------------------------------------------------------------------------------
%	PRESENTATION SLIDES
%----------------------------------------------------------------------------------------

\begin{frame}
\frametitle{Motivation}

\begin{itemize}

\item  Elliptic Curves: Important to Classical Algebraic Geometry and Number Theory

\begin{itemize}
    \item Fermat's Last Theorem
    \item Birch and Swinnerton-Dyer Conjecture
    \item Elliptic Curve Cryptography
    
\end{itemize}

\item Drinfeld Modules
\begin{itemize}
    \item Used to prove special cases of Langlands Conjectures [Drinfeld, 1974]
    \item Used in polynomial factorization algorithms over finite fields
    \begin{itemize}
        \item $[$Panchishkin, Potemine, 1989], [van der Heiden, 2004]
        \item Doliskani, Narayanan, S. give an expected $n^{3/2 + \epsilon} (\log q)^{1+o(1)} + n^{1 + \epsilon} (\log q)^{2 + o(1)}$ algorithm using Hasse invariants
    \end{itemize}
    \item Cryptography over Drinfeld modules - insecure [Scanlon, 2001]
\end{itemize}
  
  
  
  
\end{itemize}

\end{frame}



\begin{frame}
\frametitle{Background and Notation}

\begin{itemize}

\item $\mathbb{F}_q \subset \mathbb{K} \subset \mathbb{L}$ finite fields
\item $[\mathbb{L} : \F_q] = n$
\item Bit Complexity
\item Modular composition: given $F,G,H \in \F_q[x]$ of degree $d$, compute $F(G) \mod H$

\item $(d^{\theta} \log q)^{1 + o(1)}$
\item ???

%%%%%%%%%%%%%%%%%%%%
%\item $\omega_2$ a real number such that $n \times n$ and $n \times n^2$ matrices can be multiplied in $O(n^{\omega_2})$ field operations
  
  
  
  
\end{itemize}

\end{frame}


%------------------------------------------------
\section{First Section} % Sections can be created in order to organize your presentation into discrete blocks, all sections and subsections are automatically printed in the table of contents as an overview of the talk
%------------------------------------------------


%------------------------------------------------

\begin{frame}
\frametitle{Skew Polynomials}
\begin{itemize}
\item $\sigma$ a field automorphism
\item $\L[X,\sigma] := $ skew polynomials over $\L$ subject to $Xa = \sigma(a)X$ for $a \in \mathbb{L}$ 

\item Complexity of skew polynomial multiplication: $(k^{(\omega +1)/2}n^{\theta} \log q)^{1 + o(1)}$ for skew degree $k$

\item Let $\varphi: \F_q[x] \to L[X,\sigma]$
\item Given $\varphi(x)$, $c \in \F_q[x]$, $\varphi(c)$ can be computed in $(\deg(c)^{(\omega + 1)/2}n^{\theta} \log q)^{1 + o(1)} $


\end{itemize}

%\begin{problem}[Point Counting]
%Given a finite field $\mathbb{F}$ of order $q$, compute $|E(\mathbb{F})|$
%\end{problem}

\end{frame}

%------------------------------------------------

\begin{frame}{Drinfeld Modules: Preliminaries}
    \begin{itemize}
    \item $A:= \mathbb{F}_q[x]$

        \item Let $\gamma: A \to \mathbb{L} $ be a ring $\mathbb{F}_q$-morphism
        \item $\K$ the field generated by $\gamma(A)$
        %\begin{itemize}
        %    \item Since $L$ is finite, 
        %\end{itemize}
        \item $\sigma(x) := x^q$
        \item $\varphi_c := \varphi(c)$

    \end{itemize}
    
%    \begin{example}
 %   Let $\f = \f_2$, $L = \mathbb{F}_{16}$.
%    \[ \f_2[T]/(T^2 + T + 1) \cong \f_4 \subset \f_{16}\]
%    \[ \gamma : f(T) \mapsto f(T) \mod T^2 + T + 1 \xhookrightarrow{} \f_{16} \]
%    \[K = \f_4\]
%    \end{example}
\end{frame}


%-------------------------------------


%\begin{frame}\frametitle{Drinfeld Modules: Preliminaries}

%\begin{example}
%Let $\mathbb{F} = \f_2$, $L = \f_2[T]/(T^2 + T + 1)$, %$\sigma(x) = x^2$
%\[ a:= (T + 1)X^2 + TX + T + 1 \]
%\[b := X\]
%\[ab = (T + 1)X^3 + TX^2 + (T + 1)X\]
%\begin{align*}
%    ba & = X(T + 1)X^2 + XTX + XT + X)\\
%    & = (T+1)^2X^3 + T^2X^2 + (T^2 + 1)X \\
%     & = TX^3 + (T+1)X^2 + TX 
%\end{align*}
%\end{example}

%\end{frame}




%-------------------------------------

\begin{frame}
\frametitle{Drinfeld Modules}

\begin{definition}
A \textbf{Drinfeld A-Module} is a ring homomorphism $\varphi: A \to \mathbb{L}[X,\sigma]$ such that 

\begin{itemize}
    \item $\varphi(x) = \gamma(x) + a_1X + \ldots + a_rX^r$ with $a_r \neq 0$ for some $r \geq 1$
\end{itemize}
\end{definition}

\begin{itemize}
    \item The value $r$ is referred to as the \textit{rank} of the Drinfeld Module
        \item In the rank-2 case we say $\varphi = (g, \Delta)$ with $\varphi(x) = \gamma(x) + gX + \Delta X^2$
    %\item For any $a \in A$, $\phi_a := \phi(a)$
\end{itemize}

\end{frame}





%-----------------------------------

\begin{frame}
\frametitle{Point Counting}

\begin{itemize}
\item Classical problem: given an elliptic curve $E$, find the number of points over some finite field $\mathbb{F}_q$ 
\item Schoof gave the first polynomial time algorithm
\item Hasse's theorem provides a bound $ | |E(\mathbb{F}_q)| - q - 1  | \leq 2 \sqrt{q} $

\item Computing the LHS above reduces to computing the characteristic polynomial of the Frobenius endomorphism

\item Goal: Translate this to the rank-2 Drinfeld Module setting
\end{itemize}

\end{frame}


%-------------------------------------


%-------------------------------------




%--------------------------------------------------

\begin{frame}
\frametitle{Drinfeld Modules}

\begin{theorem}[Gekeler, 1991]
Let $\phi$ be a rank-2 Drinfeld Module, and let $\tau = \sigma^n$. Then there is a polynomial, the \textbf{characteristic polynonmial} of $\varphi$,  $Y^2 - aY +b \in A[Y]$ such that

\[\tau^2 -\phi_a\tau + \phi_b = 0\]
in $L[X,\sigma]$
\end{theorem}

\begin{itemize}
    %\item The \textit{Characteristic Polynomial} of $\phi$
    \item $a$ is the \textit{Frobenius Trace}, $b$ the \textit{Frobenius Norm}
    \item $\deg(a) \leq \frac{n}{2}$, $\deg(b) \leq n$
\end{itemize}

\end{frame}


\begin{frame}
\frametitle{Drinfeld Modules: The Main Problem}

\begin{problem}
Given a rank-2 Drinfeld module $\phi = (g,\Delta)$, compute its Frobenius Trace and Norm
\end{problem}

\end{frame}

%-------------------------------------------------

%\begin{frame}{}
%    \begin{example}
%$\f = \f_2$, $L = \f_{16}$, $\gamma = \textnormal{quotient by } T^2 + T + 1$, $\phi = (1,1)$

%\[b = T^4 + T^2 + 1\]
%\[a = T^2 + T\]
%\[\phi_{T}^2 = X^4 + (T^2 + T)X + T^2\]
%\[\phi_T^4 = X^8 + X^2 + T + 1\]
%\[X^8 + \phi_aX^4 +  \phi_b\]
%\[X^8 - (X^4 + X^2 + (T^2 + T + 1)X + T^2 + T)X^4 + X^8 + X^4 + X^2 (T^2 + T)X + T^2 + T + 1

%\]
%\end{example}
%\end{frame}




%-----------------------------------------------


\begin{frame}
\frametitle{Drinfeld Modules}

\begin{itemize}
\item Computing the Frobenius norm is relatively straightforward [Gekeler, 1991]
\item Computing the Frobenius trace turns out to be much harder

\begin{theorem}
One can compute the Frobenius trace of a rank 2 drinfeld module
\begin{itemize}
\item[(1)] in Monte Carlo time $(n^{1.885} \log q + n \log^2 q)^{1+o(1)}$,
  if the minimal polynomial of $\Phi_x$ has degree $n$
\item[(2)] in time $(n^{2+\varepsilon} \log q + n \log^2 q)^{1+o(1)}$, for
  any $\varepsilon > 0$
\item[(3)] in Monte Carlo time $(n^2 \log^2 q)^{1+o(1)}$
\end{itemize}
\end{theorem}

\end{itemize}


\end{frame}

\begin{frame}
\frametitle{Frobenius Trace}

\begin{itemize}
    \item $[$Gekeler, 2008] gives a straightforward approach
    \item Set:
    \begin{itemize}
        \item $b = \sum_{i=0}^n b_i T^i$, $b_i \in \mathbb{F}$
        \item $\phi_{T^i} = \sum_{j=0}^{2i}f_{i,j} X^j$, $f_{i,j} \in L$
    \end{itemize}
    \item Recall: $\tau^2 - \phi_a\tau + \phi_b = 0$
    \item $X^{2n} - \sum_{i \leq n/2}a_i\sum_{j\leq 2i}f_{i,j} X^{n + j} + \sum_{i\leq n}b_i \sum_{j \leq 2i}f_{i,j} X^j = 0 $
        \item By equating coefficients:
    \begin{itemize}
    $\sum_{0 \leq i \leq \frac{n}{2}} a_i f_{i,j-n} - \sum_{\frac{j}{2} \leq i \leq n} b_if_{i,j} = $ \begin{cases} 0 & \text{if}\ n \leq j < 2n \\ 1 & \text{if}\ j = 2n   \end{cases}
    \end{itemize}
\end{itemize}


\end{frame}


%-----------------------------------------------



%-----------------------------------------------

\begin{frame}{Frobenius Trace}

\begin{itemize}

        \item Set $\alpha_j = \sum_{\frac{j+n}{2} \leq i \leq n} b_i f_{i,j + n} + \delta_{j,2n}$
    \item Removing redundant equations gives an upper triangular system of size $\left\lfloor \frac{n}{2} \right\rfloor + 1$ in terms of $f_{i,j}$, $a_i$, $\alpha_j$
    \item Can compute $f_{i,j}$ using the recurrence $f_{i+1,j} = \gamma(T) f_{i,j} + g f_{i,j-1}^q + \Delta f_{i,j-2}^{q^2}$

\item Overall runtime cubic in $n$

\end{itemize}

%\begin{equation}
%\begin{bmatrix} f_{0,0} & f_{1,0} & \ldots & f_{\lfloor n/2 \rfloor, 0} \\
%                 0      & f_{1,2} & \ldots & f_{\lfloor n/2 \rfloor, 2}  \\
%                 0      & 0       & \ddots & \vdots                      \\
%                 \vdots  & \vdots  &  \ddots      & \vdots                       \\
%                 0  & 0 & \ldots & f_{\lfloor n/2 \rfloor, n}
%\end{bmatrix}
%\begin{bmatrix}
%a_0 \\ a_1 \\ \vdots \\ a_{\lfloor n/2 \rfloor}
%\end{bmatrix} = \begin{bmatrix} \alpha_{0} \\ \alpha_{2} \\ \vdots \\ \alpha_{n} \end{bmatrix}
%\end{equation}
    
\end{frame}

%-------------------------------------------------

\begin{frame}{Frobenius Trace}
\begin{itemize}

\item A faster algorithm based on the Hasse invariant exists when $\mathbb{L} = \mathbb{K}$
\item Narayanan gave a randomized algorithm when $q$ is odd and  CharPoly$(\varphi_x) = $ MinPoly$(\varphi_x)$
\item Includes an unjustified claim which we repair
\end{itemize}
\end{frame}

%------------------------------------------------



%------------------------------------------------

\begin{frame}{A Randomized Algorithm}

\begin{itemize}
    \item Recall: $\tau^2 + \varphi_b = \varphi_a \tau$
    \item Choose random elements $\alpha \in L$, $\ell : L \to \f$
    \item Identify $\tau \mapsto \sigma^n$, $r := \alpha + \varphi_b(\alpha) = \varphi_a(\alpha)$
    \item $a := \sum_{i=0}^{\left\lfloor \frac{n}{2} \right\rfloor}a_ix^i$
    \item For $j \geq 0$: $\ell(\varphi_x^j(r)) = \sum_{i = 0}^{\left\lfloor{\frac{n}{2}} \right\rfloor}a_i\ell(\varphi_x^{i+j}(\alpha))$
    \item For a choice of $\kappa$, we can construct a Hankel system
\end{itemize}
\[ \begin{bmatrix}\ell(\alpha) & \ell(\varphi_x(\alpha)) & \ldots & \ell(\varphi_x^{\left\lfloor n/2 \right\rfloor}(\alpha)) \\ \vdots & \vdots & & \vdots \\ 

\ell(\varphi_x^{j}(\alpha)) & \ell(\varphi_x^{1+j}(\alpha)) & \ldots & \ell(\varphi_x^{\left\lfloor n/2 \right\rfloor+j}(\alpha)) \\ \vdots & \vdots & & \vdots \\

\ell(\varphi_x^{\kappa}(\alpha)) & \ell(\varphi_x^{1 + \kappa }(\alpha)) & \ldots & \ell(\varphi_x^{\left\lfloor n/2 \right\rfloor + \kappa}(\alpha))

\end{bmatrix} \begin{bmatrix} a_0 \\ a_1 \\ \vdots \\ a_i \\ \vdots \\ a_{\left\lfloor n/2 \right\rfloor} \end{bmatrix} = \begin{bmatrix} \ell(r) \\ \ell(\varphi_x(r)) \\ \vdots \\ \ell(\varphi_x^j(r)) \\ \vdots  \\   \ell(\varphi_x^{\kappa}(r)) \end{bmatrix} \]
    
\end{frame}

%------------------------------------------------


%-------------------------------------------------


\begin{frame}{A Randomized Algorithm}

\begin{itemize}
    \item  With probability at least $(1 - \frac{n}{2q})^2$ we have that $\minpol(\{\ell(\varphi_x^i(\alpha)\}_i) = \minpol(\varphi_x)$
    \item $\frac{n-1}{2} \leq \deg \minpol(\varphi_x) \leq n$
    \item Take $\kappa = \deg \minpol(\varphi_x)$
    \item After possibly augmenting the Hankel matrix, the lemma guarantees an upper left submatrix of size at least $\left\lfloor \frac{n}{2} \right\rfloor + 1$ is invertible in almost all cases
    \begin{itemize}
        \item If $n$ is even and $\deg \minpol(\varphi_x) = \frac{n}{2}$ this may fail
        \item Compute $a_{n/2}$ directly [Gekeler, 1991]
    \end{itemize}
\end{itemize}
    
\end{frame}


%-------------------------------------------------

\begin{frame}{A Randomized Algorithm: Runtime}
    \begin{itemize}
        \item Takes $n^2 \log q$ $\f_q$ operations to compute all entries of the Hankel matrix
    \item $n$ to solve the Hankel System
   % \item Overall bit complexity 
    \end{itemize}
\end{frame}

%-------------------------------------------------

\begin{frame}{A Deterministic Algorithm}

\begin{itemize}
    \item We can instead exploit the ring-homomorphic properties of $\varphi$
    \item Suppose $ \frac{n}{2} + 1 < q$
    \item Pick a set $\{e_0, \ldots e_{\frac{n}{2}}\} \subset \mathbb{F}$
    \item Characteristic equation: $a(e_i) X^n  = X^{2n} + b(e_i) \mod \varphi_{x} - e_i $
    \item Find $a(e_i)$ and interpolate
\end{itemize}
    
\end{frame}


%-------------------------------------------------

\begin{frame}{A Deterministic Algorithm}
\begin{itemize}
    \item Want to compute $X^n \mod \varphi_{x} - e_i$
    \item Define $X^j := \nu_j + \mu_j X \mod \phi_{T} - e_i$ with $\nu_j, \mu_j \in L$
    %\item $\phi_T - e_i = \gamma_T - e_i + gX + \Delta X^2$
    %\item 
    \item We obtain the recurrences $\nu_{j+1} = -\frac{\gamma_x - e_i}{\Delta}\mu_{j}^q$ and $\mu_{j + 1} = \nu_j^q - \frac{g}{\Delta} \mu_j^q$
    
    \item Define $\alpha := -\frac{\gamma_x - e_i}{\Delta}$, $\beta := - \frac{g}{\Delta}$
    
    \item And let $M^{(q^j)} := \begin{bmatrix} 0 & \alpha^{q^j} \\ 1 & \beta^{q^j} \end{bmatrix}$
    
    \item Then we have $\begin{bmatrix} \nu_{n} \\ \mu_n  \end{bmatrix} = M M^{(q)} \ldots M^{(q^{n-1})}  \begin{bmatrix} 1 \\ 0  \end{bmatrix}$
    
    
    
\end{itemize}
    
\end{frame}

%------------------------------------------------

\begin{frame}{A Deterministic Algorithm}
\begin{itemize}
\item Recall $\begin{bmatrix} \nu_{n} \\ \mu_n  \end{bmatrix} = M M^{(q)} \ldots M^{(q^{n-1})}  \begin{bmatrix} 1 \\ 0  \end{bmatrix}$
    \item To compute $\nu_n, \mu_n$ efficiently, set $ P_j := M M^{(q)} \ldots M^{(q^j)}$
    \item We have the recurrence $P_{2j + 1} = P_{j} P_{j}^{(q^{j+1})}$
    \item If $\mu_n \neq 0$, $a(e_i) = \nu_n + \nu_n^q + \mu_n^q \beta$
    \item Otherwise $a(e_i) = \nu_n + b(e_i)$
    \item This approach can be extended to the cases where $\frac{n}{2} + 1 \geq q$ by replacing $\varphi_{x-e_i}$ with irreducibles
\end{itemize}
    
\end{frame}

%-------------------------------------------------


%-------------------------------------------------

%\begin{frame}
%\begin{example}
%Let $q = 5$, $n = 4$, $L = \mathbb{F}_5[T]/(T^4 + 4T^2 + 4T + 2)$, $\phi = (1,1)$. Then for $e_0 = 0$ we have $\alpha = 4t$, $\beta = 4$. Letting

%\[M = \begin{bmatrix}0 & 4T \\ 1 & 4 \end{bmatrix}\]

%We get:

%\[M M^{(5)} M^{(25)} M^{(125)} = \begin{bmatrix} T^3 + T^2 + 3 & T^3 + 2T + 3 \\ 2T^3 + 2T^2 + 3T + 4 & 4T^3 + 4T^2 + 4 \end{bmatrix}\]

%So $\nu_4 = T^3 + T^2 + 3$ and $\mu_4 = 2T^3 + 2T^2 + 3T + 4$ and

%\[a(0) = \nu_4 + \nu_4^5 + 4\mu_4^5 = 2\]


%\end{example}
%\end{frame}

%\begin{frame}

%\begin{example}
%Repeating for $e_1 = 1$, $e_2 = 2$ we get $a(1) = 3$ and $a(2) = 3$. We interpolate to get

%\[a = (T-1)(T-2) + 2T(T-2) + 4T(T-1)   = 2T^2 + 4T + 2.\]
%\end{example}
    
%\end{frame}

%--------------------------------------------------

\begin{frame}{Conclusion}

\begin{itemize}
    \item We provide two new algorithms for computing the characteristic polynomial of a rank-2 Drinfeld module
    \item Sub-cubic runtime in $[L:\f]$
    \item Works for all finite Drinfeld modules
\end{itemize}
    
\end{frame}

%--------------------------------------------------

%\begin{frame}
%\frametitle{References}
%\footnotesize{
%\begin{thebibliography}{99} % Beamer does not support BibTeX so references must be inserted manually as below
%\bibitem[Smith, 2012]{p1} M.B. Paterson, D.R. Stinson (2015)
%\newblock Combinatorial Characterizations of algebraic manipulation detection codes involving generalized difference families
%\end{thebibliography}
%}
%\end{frame}

%------------------------------------------------

%\begin{frame}
%\Huge{\centerline{Fin}}
%\end{frame}

%----------------------------------------------------------------------------------------

\end{document} 